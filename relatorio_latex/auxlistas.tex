%==========================================================================
% BEGIN LISTA DE SIGLAS E ACRÓNIMOS
%==========================================================================

%% Portuguese babel does not translate this environment
\renewcommand{\nomname}{Lista de Siglas e Acrónimos}

%% Text that can be shown before acronyms list
\renewcommand{\nompreamble}{<<Apresentar uma lista com todas as siglas e acrónimos utilizados durante a realização do trabalho. O formato base para esta lista deverá ser da forma como abaixo se apresenta.>>}

%% acronyms
\nomenclature[01]{\textbf{BD}}{Base de Dados}
\nomenclature[02]{DW}{Data Warehouse}
\nomenclature[03]{OLTP}{On-Line Analytical Processing}
\nomenclature[04]{...}{...}

%% Show acronyms
\printnomenclature



%==========================================================================
% END LISTA DE SIGLAS E ACRÓNIMOS
%==========================================================================


%==========================================================================
% BEGIN ANEXOS
%==========================================================================

%% Why \addchap, instead of \chapter? 
%% \addchap has no numbering but appears in table of contents.
\addchap{Anexos}

    <<Os anexos deverão ser utilizados para a inclusão de informação adicional necessária para uma melhor compreensão do relatório o para complementar tópicos, secções ou assuntos abordados. Os anexos criados deverão ser numerados e possuir uma designação. Estes dados permitirão complementar o Índice geral do relatório relativamente à enumeração e apresentação dos diversos anexos.>>
    
    %% section version of \addchap
    \addsec{Anexo 1}


%==========================================================================
% END ANEXOS
%==========================================================================